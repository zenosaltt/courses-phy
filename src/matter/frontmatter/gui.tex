\chapter*{Guida al testo}
Saremo onesti e concisi: Questi appunti sono lunghi e prolissi.
Ma alcuni aspetti di questi appunti sono
stati curati per soddisfare le esigenze di coloro che intendono solo dare rapide
occhiate alle nozioni di questo corso.

\begin{itemize}
    \item Indice: l'indice è la prima arma a portata di studenti che intendono
    consultare rapidamente gli appunti. Ci siamo impegnati al meglio per rendere
    concisi i titoli di capitoli e sezioni per raggiungere questo scopo. L'indice
    contiene collegamenti cliccabili che conducono direttamente alla pagina selezionata.

    \item Microindici: all'inizio di ogni capitolo viene collocato un microindice
    contenente la lista di macrosezioni, visibile al margine destro. Anche questi microindici sono dotati di
    collegamenti (cliccabili) interni alle rispettive pagine del testo.

    \item Box colorati: definizioni, leggi e principi notevoli sono risaltati
    da box colorati.
    
    \item Equazioni etichettate: se non evidenziate dai box, le leggi e altri
    risultati (per lo più matematici) rilevanti sono comunque numerati tra
    parentesi tonde.

    \item Appendici: vengono anche curate alcune appendici che riportano risorse
    utili di svariato genere, quali un piccolo compendio di esercizi, le leggi
    fisiche rilevanti in questo corso e altro ancora. L'indice riporta anche
    queste appendici.
\end{itemize}

Per coloro che invece manifestano maggiore interesse per questa scienza, alcuni
capitoli includono approfondimenti che possono stuzzicare conoscenze impavide (anche
se non sono di certo questi appunti a contenere tutti quelli più interessanti,
qui troverete solo un minimo assaggio). Anche queste sezioni potrebbero essere
contrassegnate nell'indice.
Gli approfondimenti non sono stati trattati durante le lezioni frontali, ma
\textit{alcuni} possono esser stati solo citati.

\section*{Sulla notazione}
Si utilizzano spesso in questo testo notazioni matematiche compatte.
Invitiamo il lettore a consultare l'appendice dei simboli per eventuali
chiarimenti.

%Un intero capitolo di approfondimento contiene argomenti
%relativi all'Elettromagnetismo, non più mostrato, come un tempo, durante
%l'anno accademico attuale (2023/2024).