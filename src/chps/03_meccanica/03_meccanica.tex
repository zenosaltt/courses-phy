\chptr{Meccanica}
\marginpar{\minitoc}

\section{Lavoro di una forza}


Il lavoro di una forza costante corrisponde al prodotto scalare\footnote{Ricordiamo
che il prodotto scalare $p$ tra due vettori \textbf{a} e \textbf{b} è un valore scalare
definito da $p = |\textbf{a}||\textbf{b}|\cos\theta_{\textbf{ab}}$, con $\theta_\textbf{ab}$
l'angolo tra i vettori.}
tra la forza e lo spostamento:
\begin{align}
W = \textbf{F}\cdot\textbf{r}
\end{align}
Il lavoro si misura in Joule (J), dove
\[ 1\text{ J} = 1\text{ Nm} = 1\text{ kg$\frac{\text{m}^2}{\text{s}^2}$} \]


Lavoro di una forza variabile durante lo spostamento:
\begin{align}
    W_{AB} = \int_{A}^{B}\textbf{F}(r)\cdot d\textbf{r}
\end{align}

\section*{app}
Si parte da $\overrightarrow{F} = m\overrightarrow{a} \Rightarrow
\overrightarrow{F} = m\frac{d\overrightarrow{v}}{dt}$. Abbiamo una traiettoria
qualsiasi, posizione descritta da $\overrightarrow{r}(t)$. Prendiamo una variazione
infinitesima della posizione $\Delta\overrightarrow{r}\to d\overrightarrow{r} =
\overrightarrow{r}_f - \overrightarrow{r}_i$.

\subsection{Lavoro di una forza}
\[ dW = \textbf{F}\cdot d\textbf{r} = |\textbf{F}||d\textbf{r}|\cos\theta \]
Può essere positivo, negativo, nullo.
\begin{itemize}
    \item \textbf{Lavoro motore} $dW > 0$
    \item \textbf{Lavoro resistente} $dW < 0$ (la forza si oppone allo spostamento)
    \item \textbf{Lavoro nullo} $dW = 0$ (la forza è ortogonale allo spostamento),
    es forza centripeta
\end{itemize}

Analisi dimensionale:
\[ \left[dW\right] = \left[\textbf{F}d\textbf{r}\right] = \left[M\frac{L}{T^2}L\right] = \left[M\left(\frac{L}{T}\right)^2\right] \]
\[ 1\text{ J} = 1\text{ Nm} = 1\text{ kg$\frac{\text{m}^2}{\text{s}^2}$} \]

Esempio cammino sentiero
\[ W_{A\to B} = \sum_{i = A}^{N = B}dW_i \to \int_{A}^{B}dW \text{ per } N\to\infty \]
\[ W_{A\to B} = \int_{A}^{B}\textbf{F}\cdot d\textbf{r} \]
Notiamo che 
\[ \textbf{P}\cdot d\textbf{r} = |\textbf{P}| (|d\textbf{r}|\cos\theta) \]
Dove abbiamo la proiezione dello spostamento sul peso P. Questo significa che
per calcolo del lavoro importa solo la variazione della quota, $dh$.
\[ W_{0\to 2000\text{ m}} = \sum_{0}^{2000}\textbf{P}\cdot d\textbf{r} = \int_{0}^{2000}mg dh = mgh_{bondone} \]

\subsubsection*{Versore}
\[ \hat{v} = \frac{\overrightarrow{v}}{|\overrightarrow{v}|} \]

\[ W = \int_{i}^{f}\overrightarrow{F}\cdot d\overrightarrow{s} = \int_{0}^{L}(F\hat{f})\cdot(ds\hat{x}) = F(\hat{f}\cdot\hat{x})\int_{0}^{L}ds = F\cos(\theta) L\]
Quindi
\[ \cos\theta = \frac{W}{FL} \]

\section{Teorema delle forze vive}
\textit{vis viva} (forza viva), quantità che viene dalle forze, che pareva avere
vita propria, in grado di trasferirsi da corpo a corpo.
Partiamo da seconda legge della dinamica
\[ \textbf{F} = m\frac{d\textbf{v}}{dt} \]
Senza giustificare le ragioni matematiche dei prossimi passaggi, ma facendoci
guidare dall'intuizione fisica, eseguiamo il prodotto scalare con $d\textbf{s}$
su entrambi i membri
\[ \textbf{F}\cdot d\textbf{s} = m\frac{d\textbf{v}}{dt}\cdot d\textbf{s} = m d\textbf{v}\cdot\frac{d\textbf{s}}{dt} \]
Notiamo che questa operazione ha permesso di ottenere un lavoro al membro di
sinistra, mentre a destra si ottiene il termine $d\mathbf{s}/dt$, che corrisponderebbe
proprio alla velocità $\textbf{v}$. Con ulteriori sviluppi, si raggiunge la
seguente equazione (il simbolo $d$ ha il significato fisico di variazione o
differenza infinitesima).
\[ \textbf{F}\cdot d\textbf{s} = m\textbf{v}\cdot d\textbf{v} = m\text{ }d\left[\frac{v^2}{2}\right] \]
Dimostriamo come sviluppare il termine $\textbf{v}\cdot d\textbf{v} = d\left[\frac{v^2}{2}\right]$.
Ricorriamo alla definizione vettoriale di prodotto scalare\footnote{Il prodotto scalare
di due vettori \textbf{a} e \textbf{b} è calcolabile anche attraverso la somma
dei prodotti tra i valori delle componenti dei due vettori: $\mathbf{a}\cdot\mathbf{b} = a_xb_x + a_yb_y + ...$}
e utilizziamo questo abuso di notazione, ma ragionevole dal punto di vista fisico:
\[ \int x \,dx =  \frac{x^2}{2} + c \Rightarrow \frac{d}{dx}\left[\frac{x^2}{2} + c\right] = x \Rightarrow \int x \,dx = \int d\left[\frac{x^2}{2}\right] \]
Quindi
\[ v_x dv_x + v_y dv_y + v_z dv_z \Rightarrow d\left[\frac{v_x^2}{2}\right] + ... = d\left[\frac{v_x^2 + ...}{2}\right] = d\left[\frac{v^2}{2}\right] \]
Riprendendo l'equazione $\mathbf{F}\cdot d\textbf{s} = m\text{ }d\left[\frac{v^2}{2}\right]$ otteniamo
\[ dW = d\left[\frac12 mv^2\right] \]
Il termine $E_K = \frac{1}{2}mv^2$ viene chiamato \textit{energia cinetica}. Quindi
\[ dW = dE_K \]
Questa equazione può essere tradotta come ``una infinitesima quantità di lavoro
corrisponde ad una variazione infinitesima dell'energia cinetica''. Ora possiamo
enunciare il teorema delle foze vive:
\vspace{8pt}
\begin{tcolorbox}[colback = red!30, colframe = red!30!black, title = {Teorema dell'energia cinetica (o delle forze vive)}]
    Quando una forza (risultante) applicata a un oggetto per un dato tratto di
    traiettoria, compie su di esso un lavoro, il risultato è una variazione del
    modulo della velocità dell'oggetto e quindi una variazione della sua energia
    cinetica. Quindi, il lavoro compiuto su un oggetto è uguale alla variazione
    della sua energia cinetica.
    \begin{align}
        W_{i\to f}^{(R)} = E_{K,f} - E_{K,i}
    \end{align}
\end{tcolorbox}
\vspace{5pt}

\noindent È necessario sottolineare alcune osservazioni:
\begin{enumerate}
    \item Il teorema presuppone che il lavoro sia dovuto all'effetto della risultante delle forze agenti sul corpo.
    \item Il lavoro è rappresentato da una variazione di energia cinetica. Possiamo descrivere dunque lo stato finale
    come \[ E_{K,f} = E_{K,i} + W_{i\to f} \] dunque se il lavoro, quindi l'energia trasferita all'oggetto, è positivo,
    l'energia cinetica aumenta e viceversa.

    \item Il teorema sposta la descrizione del sistema fisico dal piano vettoriale a quello scalare. Ovvero, partendo
    da grandezze vettoriali, abbiamo ottenuto una legge dove compaiono solamente dei numeri. Ciò rende particolarmente agevole
    l'utilizzo del teorema in svariati problemi nei quali l'analisi vettoriale può essere ostica.

    \item Il teorema è molto potente per la sua validità generale, perché non sono state fatte ipotesi sulla natura delle
    forze, se non presupponendo come vera la seconda legge della dinamica $\textbf{F} = m\textbf{a}$.
\end{enumerate}

\subsubsection*{Applicazione del teorema delle forze vive}
Supponiamo di avere un punto di massa $m$ sulla sommità di un piano inclinato di
alteza $h$ con
un angolo $\theta$ rispetto all'orizzontale. Sappiamo che la massa parte dalla
cima con velocità $v_i$ parallela alla lunghezza del piano e diretta verso la
discesa. Vogliamo trovare la velocità finale della massa. Si esprima il calcolo sia
con che senza attrito, considerando nell'ultimo caso un coefficiente di attrito
dinamico $\mu$.
\[ W = \Delta E_K \]
\[ W = \mathbf{P}\cdot\mathbf{L} = |\mathbf{P}||\mathbf{L}|\cos\left(\frac{\pi}{2} - \theta\right) = |\mathbf{P}||\mathbf{L}|\sin\theta = |\mathbf{P}|h = mgh \]
\[ \Delta E_K = \frac12mv_f^2 - \frac12mv_i^2 \]
Quindi
\[ mgh = \frac12mv_f^2 - \frac12mv_i^2 \]
\[ v_f = \sqrt{2gh + v_i^2} \]

Con attrito il peso è contrastato. L'attrito compie il suo lavoro per tutta la
lunghezza del piano fino alla fine della discesa. L'espressione della differenza
di energia cinetica è tuttavia la stessa.
\[ W = W_P - W_A = Ph - F_AL = Ph - \mu F_\perp L = mgh - \mu mg\sin\theta L \]