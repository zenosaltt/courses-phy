\chptr{Tavola dei simboli e notazioni}

\section{Sulla notazione vettoriale}
Il testo rappresenta i vettori con lettere maiuscole o minuscole,
italiche e in grassetto. Tale scelta è stata adottata in quanto
standardizzata e più pratica da impiegare su carta. Questa notazione
è equivalente a quella utilizzata abitualmente alla lavagna, dove
le lettere sono sovrastate da una freccia che punta verso destra.

\begin{center}
    $\vecsymb{v}$ equivale a $\vec{v}$
\end{center}

\noindent Se la lettera non è grassettata e il contesto di interpretazione
non è ambiguo, si intende il \textit{modulo} del vettore.

\begin{center}
    $v$ equivale a $|\vec{v}|$
\end{center}


\section{Sui separatori decimali}
In questi appunti si utilizza la notazione anglosassone per separare tra
loro le cifre dei numeri. Il punto (.) ha significato di separatore tra
parte intera e frazionaria, mentre la virgola (,) separa le cifre presenti
tra ordini di grandezza multipli di 3 nella parte intera. Ecco alcuni
esempi:

\begin{align*}
    123.456 &= \text{ centoventitré virgola quattro cinque sei}\\
    123,456 &= \text{ centoventitrémilaquattrocentocinquantasei}
\end{align*}


\section{Simboli}

\begin{align*}
    &\therefore  & \text{Quindi}\\
    &F           & \text{Forza, modulo}\\
    &\vecsymb{F} & \text{Forza, vettore}\\
    &\omega      & \text{Velocità angolare}\\
    &p           & \text{Quantità di moto, modulo}\\
    &\vecsymb{p} & \text{Quantità di moto, vettore}\\
    &\vecsymb{I} & \text{Impulso, vettore}
\end{align*}

\section{Sui pedici}
I pedici aiutano a rendere più espliciti i simboli, ma alcuni possono
aver bisogno di chiarimenti. Ogniqualvolta si tratta di quantità che
variano nel tempo, come ad esempio uno spostamento nello spazio, il
pedice $i$ indica l'istante iniziale della variazione, mentre $f$
quello finale. Nel caso dello spostamento, $\vecsymb{x}_i$ indica la
posizione iniziale (descritta dal rispettivo vettore), $\vecsymb{x}_f$
quella finale. Può capitare, per evitare ambiguità, che al posto del
pedice $i$ si utilizzi la cifra numerica 0, sempre per indicare
l'istante iniziale. In tal caso $\vecsymb{x}_i$ ha lo stesso significato
di $\vecsymb{x}_0$