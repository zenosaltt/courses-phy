\chptr{Relatività galileiana}
\marginpar{\minitoc}

Cominciamo questo capitolo ponendo al lettore una domanda:

\begin{center}
    \textit{Sei aristotelica/o oppure galileiana/o?}
\end{center}

\noindent Per coloro che la capiscono e sono in grado di rispondere:
Molto bene. Per color che non la comprendono: Siete un passo avanti
rispetto a tutti gli altri. Si può affermare tutto ciò perché, seppur
nell'età contemporanea, non è ancora stata assimilata a fondo la
relatività dei moti. E chi sbaglia in alcuni esercizi e problemi
non sono solo le persone che non hanno risposto alla domanda precedente,
ma anche chi crede di sapere cosa significhi ``moto relativo''. Perché
la realtà non è sempre come sembra. Riteniamo che chi non abbia affrontato
la relatività sia un passo avanti rispetto agli altri perché non si è
contaminati da preconcetti o conoscenze pregresse su un argomento così
delicato. Di tutti i capitoli di questi appunti, avvisiamo che proprio
questo è il più pericoloso. Studi dunque il lettore da qui a proprio
rischio e pericolo.

Curioso di leggere o studiare da qui? Allora consigliamo prima la visione
di questo frammento (10 minuti a partire dal tempo 1:43:00):

\begin{center}
    \href{https://youtu.be/0kxarmulkiA?feature=shared&t=6180}{\textcolor{blue}{\textit{Marco Paolini - ITIS Galileo}}}
\end{center}

\section{Sistemi di riferimento}
\section{Principio di relatività galileiana}
\section{Forze apparenti}