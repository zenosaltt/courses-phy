\chptr{Esercizi}

\section{Effetto centrifuga}

Un corpo è vincolato a muoversi lungo una guida
rigida (può muoversi parallelamente alla guida
ma non perpendicolarmente ad essa) che sta
ruotando intorno ad uno dei suoi estremi con
velocità angolare $\omega$ (rad/s). Inizialmente
il corpo si trova a $d_0$ m dall'asse di rotazione
ed è fermo; dopo un certo intervallo di tempo
il corpo ha raggiunto una posizione $d_f$ (m)
ed ha una velocità $v_f$ (m/s). Dati
$d_0 = 1$, $d_f = 2.5$, $v_f = 8.1$, trovare
$\omega$.
\begin{enumerate}
    \item 5.0
    \item 5.6
    \item 8.9
    \item 3.5
\end{enumerate}

\noindent \textbf{Soluzione:}
Fissiamo un sistema di riferimento
solidale con la guida. In tal modo, è come se
l'oggetto si stesse muovendo solamente in linea
retta sulla guida stessa. In tale situazione,
l'oggetto si sposta verso l'esterno per via
della forza centrifuga, una forza apparente,
perché il sistema di riferimento scelto
non è inerziale.

Ricorriamo ad una analisi energetica. L'energia
cinetica dell'oggetto viene incrementata dall'azione
della forza centrifuga; ciò suggerisce il ricorso
al teorema dell'energia cinetica:

\begin{align}
    \Delta E_K &= W_{F_c}\label{g1}\\
    K_f - K_0 &= \int_{d_0}^{d_f}F_c ds\label{g2}\\
    \frac12 m v_f^2 &= \int_{d_0}^{d_f}m\omega^2sds\label{g3}\\
    v_f^2 &= \omega^2(d_f^2 - d_0^2)\label{g4}
\end{align}

\noindent da cui il risultato cercato

\[ \omega = \frac{v_f}{\sqrt{d_f^2 - d_0^2}} \]

\noindent Eseguendo i calcoli, risulta che la risposta
(4) è quella corretta. Analizziamo più chiaramente i passaggi:
In \ref{g1} esprimiamo il teorema delle forze vive
come variazione dell'energia cinetica dell'oggetto
equivalente al lavoro compiuto dalla forza centrifuga $F_c$.
In \ref{g2} esplicitiamo la differenza tra le energie
cinetiche iniziale e finale; dall'altro membro costruiamo
invece l'integrale che permette di ottenere il lavoro
totale dalla posizione iniziale $d_0$ a $d_f$: ricorriamo
alla definizione più generale di lavoro (appunto impiegando
il calcolo integrale), perché $F_c$ varia durante il
percorso da $d_0$ a $d_f$. In \ref{g3} possiamo accorgerci
che l'energia cinetica iniziale è nulla (il testo afferma
che l'oggetto è inizialmente fermo rispetto alla guida);
al membro di destra esprimiamo invece $F_c$ in funzione
di $m$, $\omega$ e $s$, dove $s$ rappresenta la posizione
dell'oggetto a partire dal fulcro sul quale ruota la guida.
Per ricavare questa relazione
abbiamo utilizzato la definizione di forza centripeta
(che si rivela essere, in modulo, la stessa di quella
centrifuga)

\[ a_c = \omega^2 s \]

\noindent In \ref{g4} cancelliamo i termini comuni
come la massa e il fattore $\frac12$ (che compare
anche al membro di destra per via dell'integrazione
del tipo $\int xdx = x^2/2$). Notiamo poi che $\omega$
è costante, perché la guida ruota con velocità angolare
costante, e può essere ``estratto'' dall'integrale.

\noindent \textbf{Nozioni chiave:} moto circolare uniforme, sistemi
di riferimento non inerziali, energia meccanica.


\section{Rincorsa sul cuneo}
Una massa $m$ (kg) si muove inizialmente su un piano
orizzontale privo di attrito con velocità $v_0$ (m/s).
Successivamente essa sale su un piano inclinato, inizialmente
fermo rispetto al piano orizzontale, di
massa $M$ (kg), che consiste in un cuneo libero di
muoversi anch'esso sul piano. La velocità $v_0$ è
tale che $m$ si ferma esattamente sulla sommità del
cuneo, ad altezza $h$ (cm). Si supponga che non vi
siano attriti di alcun genere. Dati $m = 11.0$, $M = 47.0$
e $h = 25.0$, trovare $v_0$.

\begin{enumerate}
    \item 4.1
    \item 8.5
    \item 2.5
    \item 6.0
\end{enumerate}

\noindent \textbf{Soluzione:}


\section{\textit{Raindrops are falling on my head}}
Una goccia di pioggia di raggio $R$ (mm)
cade da una nuvola. Durante la caduta
questa risente della resistenza aerodinamica
opposta dall'aria che possiamo modellizzare
come $F = \frac12 D \rho_\text{aria} S v^2$,
dove $D$ è il coefficiente aerodinamico,
$S$ la superficie d'impatto della goccia e
$v$ la sua velocità. Si assuma la goccia
sferica e le densità di aria e acqua pari a
$\rho_\text{aria} = 1.2$ kg/m$^3$ e
$\rho_\text{H$_2$O} = 1000$ kg/m$^3$. Sotto
queste assunzioni la velocità limite della
goccia cresce fino a stabilizzarsi al
valore limite $v_f$ (m/s). Dati $D = 0.78$,
$v_f = 24.0$, trovare $R$.

\begin{enumerate}
    \item 40.4
    \item 28.1
    \item 59.2
    \item 20.6
\end{enumerate}

\noindent \textbf{Soluzione:}

\section{Valuta l'offerta}
Vostro cugino Nunzio vi dice che ha inventato una macchina termica che assorbe
calore da una fonte a 30°C e cede calore ad un pozzo a 15°C. Secondo Nunzio
l'efficienza della macchina è pari al 15\%. La comprereste?

\noindent \textbf{Soluzione:}