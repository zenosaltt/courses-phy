\chptr{Meccanica e Urti}
\marginpar{\minitoc}

Vogliamo prevedere lo stato del sistema dopo l'urto, in termini di
velocità (vettori) dei corpi coinvolti. Ci serve un \textit{sistema
isolato} e la terza legge della dinamica.

\section{Quantità di moto}
Abbiamo sempre espresso la seconda legge come
\[ \mathbf{F} = m\mathbf{a} = m\frac{d\mathbf{v}}{dt} \]
ma questa proposizione afferma che l'effetto dell'agente esterno (la forza
$\mathbf{F}$) si traduce interamente in una variazione dello stato di
moto del corpo (accelerazione $\mathbf{a}$). Si suppone quindi che la
massa sia sempre costante, anche se ciò non è sempre vero. Ad esempio,
un razzo pieno di carburante non avrà la stessa massa che aveva in partenza
una volta arrivato in orbita, quindi la forza esercitata dalla propulsione
dei motori si è tradotta non solo in una variazione dello stato di moto,
ma anche in una variaizione della massa. Non tratteremo sistemi complessi
come il razzo, ma ciò fa intuire che la seconda legge della dinamica può
essere generalizzata nella forma seguente
\begin{align}
    \mathbf{F} = \frac{d}{dt}(m\mathbf{v}) = \frac{d\mathbf{p}}{dt}
\end{align}
Dove la quantità $\mathbf{p}$ prende il nome di \textit{quantià di moto},
definita come
\begin{align}
    \mathbf{p} \eqdef m\mathbf{v}
\end{align}
Come dice il termine, la quantità di moto descrive il moto dei corpi
sulla base della velocità di una massa più la massa stessa, a differenza
di quanto accade nello studio cinematico del moto, dove solo le variazioni
dello stato di moto contano, slegate da cause (forze) e materia (massa).

\section{Il fenomeno dell'urto}
\vspace{8pt}
\begin{tcolorbox}[colback = yellow!30, colframe = yellow!30!black, title = {Urto}]
    Un urto è un'interazione tra corpi, nella quale
\end{tcolorbox}
\vspace{5pt}

Un classico esempio che unisce nozioni su urti e quantità di moto è il
tavolo da biliardo. Supponiamo di avere due palle, 1 e 2, sul tavolo in moto
rettilineo
uniforme e in rotta di collisione tra loro; ovvero, sappiamo con certezza
che la loro traiettoria si intersecherà e che tale punto verrà raggiunto
da entrambi i corpi nel medesimo istante di tempo. L'esprerienza ci permette
di concludere che, passata la \textit{zona d'urto}, le palle non procederanno
sulle stesse rette dei moti precedenti, ma devieranno.
Dobbiamo fare alcune assunzioni fondamentali. Innanzitutto, supporremo che
nessun altro agente agirà sul sistema appena descritto (aiuterebbe immaginare
due palle che vagano nello spazio profondo).
Immaginiamo l'intervallo temporale nel quale le due palle saranno a contatto
tra loro, collidendo: entrambe eserciteranno una forza sull'altra e aiutati
dalla terza legge della dinamica sappiamo che
\[ \textbf{F}_{1 \to 2} = -\textbf{F}_{2 \to 1} \]
cioè l'applicazione di una forza su una palla determina una forza identica
in modulo e direzione, ma verso opposto e applicata sull'altra palla.
Sviluppiamo l'equazione sfruttando la definizione di quantità di moto,
ricordando che una forza $x \to y$ determina una variazione dello stato di
moto di $y$.
\[ \frac{d\mathbf{p}_2}{dt} = -\frac{d\mathbf{p}_1}{dt} \]
Ricorrendo agli usuali abusi di notazione matematica, ma ragionevoli da
un punto di vista fisico, semplifichiamo l'intervallo di tempo infinitesimale
del differenziale:
\begin{align*}
    &d\mathbf{p}_2 = -d\mathbf{p}_1\\
    &d[\mathbf{p}_1 + \mathbf{p}_2] = 0\\
    &d\mathbf{p}_\text{tot} = 0
\end{align*}
Abbiamo mostrato, in anticipo, che per un sistema di due corpi come le palle
da biliardo, assumendo che non agiscano forze esterne, la quantità di moto
totale del sistema si conserva. Come l'energia meccanica, possiamo dunque
concludere che un urto ideale non modifichi la quantità di moto del sistema.
Un'altra supposizione importante che è stata
sottointesa è la seguente: le distanze (eventuali variazioni della forma
degli oggetti) e i tempi (intervalli di tempo nei quali avviene il contatto
o più generalmente l'azione delle forze generate nella collisione) tipici dell'urto sono
molto piccoli e trascurabili rispetto a quelli normalmente osservabili al
di fuori dell'urto.
Con tutte queste assunzioni possiamo descrivere il
sistema attraverso la seguente approssimazione (dove $m$ indica la massa
di una palla), che consente di descrivere lo stato del sistema prima o
dopo l'urto, \textit{iniziale} e \textit{finale}.
\[ m_1v_{1,i} + m_2v_{2,i} = m_1v_{1,f} + m_2v_{2,f} \]

\section{Conservazione}
Come l'energia, nella storia della fisica si è sempre pensasto al moto
come una quantità che potesse essere trasferita da un corpo ad un altro,
fatto ragionevole ben giustificato dall'esperienza: basti pensare al
gioco del biliardo.
Come già dedusse Cartesio, la ``macchina dell'universo'', assimilata ad
un orologio, non può continuare a funzionare senza che una qualche quantità
si conservi. Egli stesso fu tra i primi a proporre il prodotto massa-velocità
come misura di tale quantità: due carri identici che viaggiano a velocità differenti
hanno chiaramente quantità di moto diverse, ma una palla di cannone
racchiude in sé maggior moto rispetto ad un sasso lanciato alla stessa
velocità. Queste quantità sono presenti negli oggetti secondo distribuzioni
differenti e variabili nel tempo, ma nel complesso esse non possono che
sommarsi sempre nella medesima quantità; se l'universo è un sistema chiuso,
e lo si può supporre per definizione, allora la quantità di moto non può
sparire o comparire, ma può trasferirsi tra gli oggetti al suo
interno, trasformarsi.

\subsection{Forze interne ed esterne}
Approfondiamo il concetto di \textit{sistema di punti materiali} e studiamone
uno contentente un certo numero di punti materiali $N$. Immaginiamo che
tra questi punti agiscano forze di varia natura: repulsive elettriche,
attrattive gravitazionali ecc.; inoltre, supponiamo che vengano applicate
altre forze dall'esterno di questo sistema di punti materiali\footnote{Una
immagine esplicativa è il polmone, dove le molecole dell'aria formano i
punti materiali e i muscoli del torace sono gli agenti esterni.}.
Possiamo suddividere le forze in gioco in due insiemi:
\begin{enumerate}
    \item \textbf{Forze interne}: le forze che i punti esercitano gli uni
    sugli altri. Forze che descrivono l'interazione tra i punti.
    \item \textbf{Forze esterne}: le forze che l'ambiente esterno esercita
    sul sistema, l'insieme di punti.
\end{enumerate}
Ogni punto $i$-esimo sarà sottoposto ad una certa forza totale, o risultante, derivante
dalla somma/sovrapposizione di tutte le forze precedentemente descritte
\begin{align*}
    \mathbf{R}_i = m_i\mathbf{a}_i\\
    \mathbf{R}^{(E)}_i + \mathbf{R}^{(I)}_i = m_i\mathbf{a}_i
\end{align*}
Definiamo le risultanti delle forze esterne, ed interne agenti sul punto
$i$-esimo:
\begin{align*}
    \mathbf{R}^{(E)}_i &\eqdef \sum_{k} \mathbf{F}^{(E)}_{k \to i}\\
    \mathbf{R}^{(I)}_i &\eqdef \sum_{j} \mathbf{F}^{(I)}_{j \to i} \qquad j \not = i
\end{align*}
Supponiamo che un punto non eserciti alcuna forza su se stesso.
La risultante di tutte le forze in gioco sarà
\[ \mathbf{R} = \sum_{i} \mathbf{R}_i \]
In tale somma, concentriamoci sulla risultante delle forze interne:
\[ \mathbf{R}^{(I)} = \sum_{i} \mathbf{R}^{(I)}_i = \sum_{i} \sum_{j} \mathbf{F}^{(I)}_{j \to i} \]
In questa somma, supponiamo che non vi siano forze agenti su un corpo
ed esercitate dal corpo stesso, dunque poniamo $\mathbf{F}^{(I)}_{j \to i} = \overrightarrow{0} \quad \forall j = i$.
Sappiamo che vale la terza legge della dinamica, dunque
$\mathbf{F}^{(I)}_{j \to i} = -\mathbf{F}^{(I)}_{i \to j}$. Ma allora
\begin{align}
    \mathbf{R}^{(I)} = \sum_{i} \sum_{j} \mathbf{F}^{(I)}_{j \to i} = \overrightarrow{0}\label{interne}
\end{align}
Abbiamo appena dimostrato, grazie all'ipotesi della terza legge,
che, in un sistema di punti materiali, la risultante delle forze interne
è nulla.

\subsection{Centro di massa}
Dall'Equazione \ref{interne} possiamo dedurre che la risultante delle
forze, interne ed esterne, coinvolte in un sistema di punti materiali
è determinata solamente dalle forze esterne.
\[ \mathbf{R} = \mathbf{R}^{(E)} = \sum_{i} \mathbf{R}^{(E)}_i = \sum_i m_i\mathbf{a}_i \]
Dalla precedente equazione, si può ricavare un'interessante definizione:
\[ \sum_i m_i\mathbf{a}_i = \sum_i m_i \frac{d^2\mathbf{x}_i}{dt^2} = \left(\sum_i m_i\right) \frac{d^2}{dt^2}\left[ \frac{\sum_i m_i\mathbf{x}_i}{\sum_i m_i} \right] \]
Abbiamo ottenuto un termine molto interessante, un artificio matematico
che ha interpretazioni e applicazioni piuttosto importanti: \textit{il
centro di massa}.
\begin{align}
    \mathbf{x}_\text{CM} \eqdef \frac{\sum_i m_i\mathbf{x}_i}{\sum_i m_i}
\end{align}
Dalla definizione è banane\footnote{Esattamente, banane. Ci scusiamo per questo scherzo di nevrosi.} ricavare la velocità e l'accelerazione del
centro di massa. Riprendendo le equazioni precedenti e ponendo $M = \sum_i m_i$,
possiamo concludere che
\[ \mathbf{R}^{(E)} = M\frac{d^2\mathbf{x}_\text{CM}}{dt^2} = M\mathbf{a}_\text{CM} \]

\subsection{La legge di conservazione della quantità di moto}
In un sistema isolato, non si rilevano forze esterne. Allora
\[ \mathbf{R}^{(E)} = \overrightarrow{0} \quad \therefore \quad M\mathbf{a}_\text{CM} = 0 \quad \therefore \quad \mathbf{a}_\text{CM} = 0 \]
ovvero, la velocità del centro di massa è costante, quindi la
quantità di moto del centro di massa è costante e si conserva in
un sistema isolato. Vale dunque la seguente in un sistema isolato:
\begin{align}
    \frac{d}{dt}\mathbf{p}_\text{tot} = 0
\end{align}
Tale risultato ha conseguenze di portata non trascurabile. Si consideri
infatti il problema esposto nella prossima sottosezione.

\subsubsection*{Punto di collisione}
Due magneti di massa $m$ e $5m$ sono mantenuti ad una distanza fissa tra
di loro. Una volta rimossi i vincoli, i due magneti si attraggono fino a
schiantarsi. Sapendo che il primo magnete si trovava in posizione $x_1 = 0$
mentre il secondo in $x_2 = 8\text{ cm}$, si intende individuare la posizione
della collisione.

Alla luce della legge di conservazione della quantità di moto, sappiamo
che nella situazione iniziale la quantità di moto totale del sistema è
nulla, in quanto i due magneti sono mantenuti fermi.
\[ \mathbf{p}_i = 0 \]
Il problema non fa alcun cenno all'azione di forze esterne, dunque possiamo
supporre che il sistema sia isolato e che dunque la quantità di moto
iniziale si conserverà anche subito prima dell'urto. Vale allora:
\[ \mathbf{p}_i = m\mathbf{v}_{i,\text{tot}} = 0 \Rightarrow \mathbf{v}_{i,\text{tot}} = 0 \]
Ma sappiamo anche che la velocità totale è rappresentato dal centro di massa
del sistema. Ciò significa che se il centro di massa era ``fermo''
inizialmente, rimarrà tale anche poco prima dell'urto, grazie alla legge
di conservazione della quantità di moto. Basta dunque trovare la posizione
del centro di massa e il gioco è fatto.
\[ x_\text{CM} = \frac{mx_1 + 5mx_2}{m + 5m} = \frac{x_1 + 5x_2}{6} = \frac{5}{6}x_2 \]


\section{Impulso}
\[ \left\langle \mathbf{F} \right\rangle = \mathbf{F}_\text{impulsiva} \simeq \frac{\Delta\mathbf{p}}{\Delta t} \]
\[ \Delta\mathbf{p} = \mathbf{F}_\text{impulsiva}\Delta t \]
Deformazioni per tempi brevi, tempo di contatto.
\[ \Delta\mathbf{p} = m\mathbf{v}_f - m(\overrightarrow{0}) \qquad F_\text{imp} = \frac{mv_f}{T} \]


\section{Urti elastici}
In un urto elastico, si conservano la quantità di moto e l'energia cinetica.
Un sistema contenente un certo numero di corpi può dunque essere descritto come segue:
\begin{align*}
    \begin{cases}
        \sum_j m_j\mathbf{v}_{0,j} = \sum_j m_j\mathbf{v}_{f,j} \qquad \text{conservazione della quantità di moto}\\
        \sum_j m_j\mathbf{v}_{0,j}^2 = \sum_j m_j\mathbf{v}_{f,j}^2 \qquad \text{conservazione dell'energia cinetica}
    \end{cases}
\end{align*}
Nella realtà comunemente osservabile, urti pressoché elasti avvengono tra
oggetti che rimbalzano tra loro dopo un urto, anche se in realtà l'energia
cinetica viene inevitabilmente dissipata in altre forme (suono e calore), rendendo
l'urto stesso non del tutto elastico. Sono idealmente elastici urti tra
particelle.

\section{Urti anelastici}
Negli urti anelastici, vale sempre la legge di conservazione della quantità
di moto. Supponiamo di avere due masse che viaggiano ad una certa velocità
e in rotta di collisione tra loro. Dopo l'urto, le due masse rimangono
in contatto. Supponendo che le due masse attaccate formino un unico corpo, Vale:
\[ \mathbf{p}_{1,i} + \mathbf{p}_{2,i} = \mathbf{p}_{3,f} \]
Dal punto di vista energetico, però, accade qualcosa di diverso. Osserviamo
l'energia cinetica:
\[ E_{K,i} = E_{K1, i} + E_{K2, i} = \frac12 m_1\mathbf{v}_{1,i}^2 + \frac12 m_2\mathbf{v}_{2,i}^2\]
Supponiamo di porre gli oggetti nel sistema di riferimento del centro di massa
e di osservare esternamente la situazione (poniamo $\mathbf{v} = \mathbf{v}_\text{CM} + \mathbf{v}_\text{scm}$):
\[ E_{K,i} = \frac12 m_1(\textbf{v}_\text{CM} + \mathbf{v}_{1,i,\text{scm}})^2 + \frac12 m_2(\textbf{v}_\text{CM} + \mathbf{v}_{2,i,\text{scm}})^2 = \frac12(m_1 + m_2)\mathbf{v}_\text{CM}^2 + E'_{K,i} \]
dove abbiamo isolato i termini nei quali compare la velocità del centro di massa
elevata al quadrato, per poi condensare l'energia rimanente in $E'_{K,i}$,
la cui forma non è per noi rilevante. Se invece calcoliamo l'energia cinetica
finale,
\[ E_{K,f} = \frac12 (m_1 + m_2)\mathbf{v}_{3,f}^2 = \frac12(m_1 + m_2)\mathbf{v}_\text{CM}^2 \]
è immediato infatti notare che, data la conservazione della quantità di moto,
il corpo finale comprensivo di entrambe le massse non potrà che muoversi
con velocità identica a quella del centro di massa.
Abbiamo dunque mostrato che in generale l'energia cinetica non si conserva nell'urto.
In particolare, essa diminuisce sempre in situazioni reali.

\begin{align}
    E_{K,f} \leq E_{K,i}
\end{align}

\noindent Il motivo di tale fenomeno può essere spiegato attraverso numerose
interpretazioni, che possono dipendere dal sistema analizzato. In generale,
la perdita di energia cinetica è dovuta al costo per mantenere unite le
masse dopo l'urto, oppure, come accade sempre nella realtà, quell'energia
viene dissipata in calore, deformazioni permanenti dei materiali, suono e
così via.