\chptr{Termodimaica}

\section{Introduzione}
D'ora in poi, il ruolo dell'energia sarà ancora più importante. In precedenza
abbiamo dimostrato che la differenza di energia meccanica di un corpo corrisponde
al lavoro totale delle forze non conservative.

\[ W^\text{NC} = \Delta E \]

\noindent In altri termini, l'energia che un corpo possiede viene persa o acquistata
nel caso in cui vi siano forze non conservative che compiano lavoro effettivo. Ma
quindi l'unico modo di ``parlare'', ovvero scambiare energia, con l'universo è quello
di compiere lavoro? In realtà no, come vedremo parlando di \textit{calore}, la forma
di energia più ``disordinata'' che conosciamo.

Ci occuperemo di sistemi contenenti un numero di costituenti nell'ordine di

\[ N_A = 6 \cdot 10^{23} \]

\noindent ossia il numero di Avogadro. Questi costituenti possono scambiare energia
attraverso ciò che definiremo calore.

\subsection*{Sistemi termodinamici}
Le nostre trattazioni avranno per oggetto i \textit{sistemi termodinamici}, per
definizione contenuto in un \textit{ambiente esterno}. Ambiente esterno e il suo
contenuto costituiscono l'\textit{universo}, per definizione l'unico sistema che
non è contenuto in un altro ambiente esterno.
Altro oggetto di nostro interesse per lo studio di questi sistemi sono le
\textit{trasformazioni termodinamiche}, ovvero processi nei quali avvengono
scambi di energia.

Possiamo classificare i sistemi termodinamici sulla base di due criteri: capacità
di scambiare materia e capacità di scambiare energia.

\subsection*{Variabili termodinamiche}
In cinematica utilizziamo variabili spazio-temporali per descrivere i nostri
sistemi di punti materiali. Fare questo per un numero di punti o costituenti
dell'ordine di $10^{23}$ è di fatto proibitivo. In termodinamica si utilizzano
invece grandezze diverse, dette \textit{variabili (o coordinate) termodinamiche}.
Le distinguiamo in

\begin{itemize}
    \item \textbf{Grandezze intensive}: si possono misurare localmente,
    indipendentemente dall'estensione dell'oggetto che ne è caratterizzato.
    Sono grandezze intensive la \textit{pressione} e la \textit{temperatura}.

    \item \textbf{Grandezze estensive}: è necessario considerare l'oggetto nel
    suo complesso, non ha senso o non si può misurare in un punto locale arbitrario.
    Sono grandezze estensive il \textit{volume} e la \textit{massa}.
\end{itemize}

\noindent Come in cinematica impiegavamo un sistema di assi cartesiani come
forma di rappresentazione del sistema di punti materiali in moto nelle dimensioni,
anche in termodinamica utilizziamo il medesimo strumento, ma utilizzando le
coordinate termodinamiche. Una trasformazione termodinamica consiste in uno
spostamento tra due punti $A$ e $B$ su questo nuovo piano cartesiano e assumeremo
sempre, affinché la nostra teoria funzioni, che $A$ e $B$ siano situazioni, stati, di
\textit{equilibrio termodinamico}. Nel mezzo non vi è garanzia di equilibrio.

Un sistema si dice in equilibrio dinamico se esso rispetta i seguenti equilibri:
\begin{enumerate}
    \item \textbf{Equilibrio meccanico}: lo stato non è sottoposto a forze totali non nulle,
    in qualsiasi coppia delle sue parti.

    \item \textbf{Equilibrio chimico}: non esiste alcuna reazione chimica tra una qualsiasi
    coppia di parti dello stato.

    \item \textbf{Equilibrio termico}: per ogni coppia di parti, la temperatura è la stessa.
\end{enumerate}

\noindent Per ``parti'' di uno stato o di un sistema intendiamo sottoinsiemi
abbastanza piccoli rispetto al sistema originale ma allo stesso tempo sufficientemente
grandi affiché gli strumenti della termodinamica funzionino.