\chptr{Termodimaica}
\marginpar{\minitoc}

\section{Introduzione}
\section{Principio zero}
\section{Principio primo}
\section{Gas ideali}
\section{La teoria cinetica dei gas}
\section{Le trasformazioni termodinamiche}
\section{Macchine}
\section{Principio secondo}
\section{Studi di Carnot}
\section{Studi di Clausius}

\section{Entropia}
Si consideri una trasformazione termodinamica \textit{reversibile} qualsiasi.
Per ipotesi di reversibilità, l'integrale di Clausius calcolato sulla curva $\gamma$
di tale trasformazione nel piano pressione-volume è nullo.

\[ \oint_\gamma \frac{dQ}{T} = 0 \]

\noindent Selezioniamo due punti, $A$ e $B$, distinti su questo ciclo. Scomponiamo
dunque l'integrale nei due percorsi, sempre reversibili, $\alpha$ e $\beta$.

\[ \int_{A,\alpha}^{B} \frac{dQ}{T} + \int_{B,\beta}^{A} \frac{dQ}{T} = 0 \]

\noindent Trattandosi di un ciclo reversibile, vale la proprietà antisimmetrica
dell'integrale. Giungiamo dunque alla seguente conclusione:

\[ \int_{A,\alpha}^{B} \frac{dQ}{T} = \int_{A,\beta}^{B} \frac{dQ}{T} \]

\noindent Dunque, qualsiasi sia il percorso tra $A$ e $B$, l'uguaglianza
precedente vale sempre per percorsi reversibili. Vi è dunque una dipendenza
di una certa quantità dai soli stati iniziale e finale della trasformazione.
Definiamo dunque, in modo simile all'energia potenziale studiata in meccanica,
la \textit{differenza di entropia tra due stati termodinamici}.

\begin{align}
    \Delta S_{AB} = S_B - S_A = \stackrel{\text{def}}{=} \int_{A,\text{rev}}^{B} \frac{dQ}{T}
\end{align}



\section*{Introduzione}
D'ora in poi, il ruolo dell'energia sarà ancora più importante. In precedenza
abbiamo dimostrato che la differenza di energia meccanica di un corpo corrisponde
al lavoro totale delle forze non conservative.

\[ W^\text{NC} = \Delta E \]

\noindent In altri termini, l'energia che un corpo possiede viene persa o acquistata
nel caso in cui vi siano forze non conservative che compiano lavoro effettivo. Ma
quindi l'unico modo di ``parlare'', ovvero scambiare energia, con l'universo è quello
di compiere lavoro? In realtà no, come vedremo parlando di \textit{calore}, la forma
di energia più ``disordinata'' che conosciamo.

Ci occuperemo di sistemi contenenti un numero di costituenti nell'ordine di

\[ N_A = 6 \cdot 10^{23} \]

\noindent ossia il numero di Avogadro. Questi costituenti possono scambiare energia
attraverso ciò che definiremo calore.

\subsection{Sistemi termodinamici}
Le nostre trattazioni avranno per oggetto i \textit{sistemi termodinamici}, per
definizione contenuto in un \textit{ambiente esterno}. Ambiente esterno e il suo
contenuto costituiscono l'\textit{universo}, per definizione l'unico sistema che
non è contenuto in un altro ambiente esterno.
Altro oggetto di nostro interesse per lo studio di questi sistemi sono le
\textit{trasformazioni termodinamiche}, ovvero processi nei quali avvengono
scambi di energia.

Possiamo classificare i sistemi termodinamici sulla base di due criteri: capacità
di scambiare materia e capacità di scambiare energia.

\subsection*{Variabili termodinamiche}
In cinematica utilizziamo variabili spazio-temporali per descrivere i nostri
sistemi di punti materiali. Fare questo per un numero di punti o costituenti
dell'ordine di $10^{23}$ è di fatto proibitivo. In termodinamica si utilizzano
invece grandezze diverse, dette \textit{variabili (o coordinate) termodinamiche}.
Le distinguiamo in

\begin{itemize}
    \item \textbf{Grandezze intensive}: si possono misurare localmente,
    indipendentemente dall'estensione dell'oggetto che ne è caratterizzato.
    Sono grandezze intensive la \textit{pressione} e la \textit{temperatura}.

    \item \textbf{Grandezze estensive}: è necessario considerare l'oggetto nel
    suo complesso, non ha senso o non si può misurare in un punto locale arbitrario.
    Sono grandezze estensive il \textit{volume} e la \textit{massa}.
\end{itemize}

\noindent Come in cinematica impiegavamo un sistema di assi cartesiani come
forma di rappresentazione del sistema di punti materiali in moto nelle dimensioni,
anche in termodinamica utilizziamo il medesimo strumento, ma utilizzando le
coordinate termodinamiche. Una trasformazione termodinamica consiste in uno
spostamento tra due punti $A$ e $B$ su questo nuovo piano cartesiano e assumeremo
sempre, affinché la nostra teoria funzioni, che $A$ e $B$ siano situazioni, stati, di
\textit{equilibrio termodinamico}. Nel mezzo non vi è garanzia di equilibrio.

Un sistema si dice in equilibrio dinamico se esso rispetta i seguenti equilibri:
\begin{enumerate}
    \item \textbf{Equilibrio meccanico}: lo stato non è sottoposto a forze totali non nulle,
    in qualsiasi coppia delle sue parti.

    \item \textbf{Equilibrio chimico}: non esiste alcuna reazione chimica tra una qualsiasi
    coppia di parti dello stato.

    \item \textbf{Equilibrio termico}: per ogni coppia di parti, la temperatura è la stessa.
\end{enumerate}

\noindent Per ``parti'' di uno stato o di un sistema intendiamo sottoinsiemi
abbastanza piccoli rispetto al sistema originale ma allo stesso tempo sufficientemente
grandi affiché gli strumenti della termodinamica funzionino.

\section*{Calore e scambi}
\section*{Gas ideali}
\subsection{Leggi dei gas ideali}
\subsection{Lavoro di un gas ideale}

\section*{La teoria cinetica dei gas}
Finora abbiamo affrontato la termodinamica parlando spesso di gas.
Lo abbiamo fatto inoltre da un punto di vista macroscopico, ovvero
utilizzando variabili termodinamiche (pressione, volume, temperatura).
Tuttavia, vogliamo ora tentare di costruire un modello che permetta
di spiegare cosa accade a livello microscopico e che sia allo stesso
tempo coerente con le leggi mostrate fino ad ora.
Per giungere a tale obbiettivo, ovvero collegare macroscopico e mocroscopico,
l'unico strumento che abbiamo a diposizione in questo corso è la meccanica.
Questa è la via che porta alla cosiddetta \textit{teoria cinetica dei
gas}.

\subsection*{Ipotesi e presupposti della teoria cinetica}
La teoria cinetica dei gas semplifica molto ciò che veramente accade
nella realtà fisica, ma è una approssimazione piuttosto buona e
soddisfacente. In particolare, chiariamo che la teoria si basa sulle
seguenti ipotesi:
\begin{itemize}
    \item Oggetto delle osservazioni è un gas \textit{ideale} (o
    perfetto) che si trova all'interno di un contenitore con volume
    $V$.

    \item Il gas è ideale nel senso che esso è costituito da
    particelle infinitamente piccole rispetto al contenitore nel
    quale sitrovano e alle distanze che le separano; tali particelle
    sono inoltre tutte uguali, compresa massa e altre proprietà.

    \item Il gas ha una densità bassa nell'ordine in cui le interazione
    tra le sue particelle è minimo o possibilmente nullo (non avvengono
    urti, interazioni gravitazionali o di qualsiasi altra natura).

    \item Gli unici urti di interesse teorico, non trascurabili, sono
    quelli con le pareti del contenitore, perfettamente elastici (la
    massa delle particelle è infinitamente piccola rispetto a quella
    delle pareti del contenitore).
\end{itemize}

\subsection*{Logica della teoria}
Per semplificare i calcoli, supponiamo che il contenitore sia un
cubo con lati di lunghezza $L$. Si consideri una sola particella
di gas (massa $m$) che sta per urtare la parete destra e la componente $x$
della sua velocità ($v_x$). La quantità di moto iniziale di tale particella
è

\[ \mathbf{p}_{x,i} = m\textbf{v}_{x,i} \]

Avviene un urto elastico con la parete destra, dunque la velocità finale
della particella è uguale in modulo a quella iniziale, ma con verso
opposto:

\[ \mathbf{p}_{x,f} = m\textbf{v}_{x,f} = -m\textbf{v}_{x,i} \]

La variazione della quantità di moto della particella equivale dunque
a

\[ \Delta\mathbf{p}_x = -2mv_x\hat{x} \]

Utilizziamo d'ora in poi i moduli dei vettori. Sappiamo che tale
variazione nella quantità di moto della particella può essere
interpretata come un impulso esercitato dalla parete; dunque la
parete esercita una forza sulla particella, ma come calcolarla?
Diventa necessario trovare un intervallo temporale entro il quale
agisce l'impulso per ricavare l'intensita della forza.
Semplicemente notiamo che la velocità della particella non varia,
per via degli urti elastici, e dunque essa rimbalza ripetutamente
a destra e sinistra nel cubo. Dunque, la particella impiega un tempo

\[ \Delta t = \frac{2L}{v_x} \]

Per lasciare la parete e tornarvici per un altro urto. In media nel
tempo, dunque, la forza impulsiva esercitata dalla parete destra
sulla particella equivale a

\[ F_x = \left|\frac{\Delta p_x}{\Delta t}\right| = 2mv_x \cdot \frac{v_x}{2L} = \frac{mv_x^2}{L} \]

Dalla definizione di pressione, otteniamo la pressione esercitata
dalla particella che rimbalza tra due pareti opposte (di superficie
$S = L^2$):

\[ p_x = \frac{F_x}{S} = \frac{mv_x^2}{L^3} = \frac{mv_x^2}{V} \]

Lo stesso ragionamento vale per tutte le componenti spaziali.
Macroscopicamente, osserviamo che la pressione è una quantità intensiva,
che non dipende dalla località della misurazione. Dunque tutte
le dimensioni spaziali sono equivalenti e possiamo effettuare la
seguente semplificazione:

\[ p_x = p_y = p_z = p \]

Unendo la legge dei gas perfetti al risultato ottenuto precedentemente,

\[ pV = mv_x^2 \]

Ma ciò riguarda solamente una singola particella. Sommiamo dunque
tutti i contributi:

\[ F_{\text{tot}, x} = \sum_i F_{x,i} = \sum_i \frac{mv_{x,i}^2}{L} = \frac{m}{L}\sum_iv_{x,i}^2 \]

In generale le velocità delle particelle potrebbero non essere tutte
uguali. Supponendo che il gas sia costituito da $N$ particelle, possiamo
determinare la loro velocità quadratica media

\[ F_{\text{tot},x} = N\frac{m}{L}\overline{v_x^2} \]

Dunque

\[ pV = Nm\overline{v_x^2} \]

Sappiamo che $v^2 = v_x^2 + v_y^2 + v_z^2$, che vale anche per la
velocità quadratica media $\overline{v^2}$. Ma trattando delle medie,
i contributi di tutte le dimensioni sono tra loro equivalenti:

\[ \overline{v_x^2} = \overline{v_y^2} = \overline{v_z^2} \]

Da cui

\[ \overline{v^2} = \overline{v_x^2} + \overline{v_y^2} + \overline{v_z^2} = 3\overline{v_x^2} \]

Allora

\[ pV = Nm\frac{\overline{v^2}}{3} = \frac{2}{3}N\overline{E_K}\]

Dove abbiamo esplicitato l'energia cinetica media della singola particella.
Per andare più in profondità, possiamo notare che la teoria può
portarci a concludere che

\[ N\overline{E_K} = E_{int} = U \]

Abbiamo dunque collegato la nostra teoria da un lato e gli esperimenti
dall'altro:

\[ pV = Nk_BT = \frac{2}{3}U \]

Possiamo inoltre osservare dalla precedente che

\[ \overline{E_K} = \frac{3}{2}k_BT \]

Che esprime la relazione tra energia cinetica media di una particella
e la temperatura. In un certo senso, dunque, la temperatura \textit{è}
proprio l'energia cinetica media delle particelle.

\subsection*{Risultati della teroria}
Il risultato più importante della teoria cinetica è l'unione tra
il mondo macroscopico osservato sperimentalmente e il mondo microscopico
modellato teoricamente. In particolare:
\begin{itemize}
    \item La teoria cinetica offre una spiegazione dell'origine della
    pressione esercitata da un gas in un contenitore.

    \item La teoria cinetica mostra che la temperatura è strettamente
    legata all'energia cinetica media delle particelle. Intuitivamente,
    la temperatura è interpretabile proprio come l'agitazione media
    delle particelle, per l'appunto l'energia del loro movimento.

    \item La teoria cinetica costruisce un modello generalizzabile
    a sistemi di costituenti non necessariamente monoatomico-puntiformi,
    come approfondito nella prossima sezione.
\end{itemize}

\subsubsection*{Gradi di libertà}
Il fattore $1/2$ deriva dalla definizione di enrgia cinetica, ma
il numero $3$ invece? Esso dipende dal numero di dimensioni entro
le quali le particelle possono muoversi, ovvero i \textit{gradi di
libertà}. Se compissimo lo studio da capo, costringendo però le particelle
a giacere su un piano, i gradi di libertà sarebbero solo due e il fattore
moltiplicativo
dell'energia cinetica media sarebbe diverso. Ciò è dovuto al fatto
che ogni grado di libertà permette alla particella di muoversi su
un'altra dimensione e dunque di aggiungere un contributo in più alla
propria energia cinetica. In generale, dunque, ogni grado di
liberta comporta un contributo energetico di $\frac12k_BT$ e
dunque, per $l$ gradi di libertà, si ottiene

\[ \overline{E_K} = \frac{l}{2}k_BT \]

I gradi di libertà possono crescere all'aumentare della complessità
della particella di gas. Oltre alle tre dimensioni, una molecola
biatomica (quindi non puntiforme come abbiamo sempre supposto finora)
può anche ruotare su due assi\footnote{Due assi sono sufficienti a
coprire tutte le rotazioni nelle tre dimensioni}; i due atomi possono
poi vibrare intorno alla loro ``sede'' nella molecola, dunque anche
questa energia deve essere presa in considerazione. In totale, una
molecola biatomica ideale può avere $l = 6$ gradi di libertà.


\section*{Trasformazioni termodinamiche}
Abbiamo mostrato che il lavoro di un gas ideale (ovvero a bassa densità)
è legato al volume e alla temperatura dalla seguente relazione:

\[ W_{A\to B} = \int_{A,\gamma}^{B}p(V)dV \]

\noindent Notare che è sempre necessario specificare il ``percorso'' $\gamma$
che il gas intraprende nel piano pressione-volume (e temperatura),
perchè in generale il lavoro dipende da tale percorso. Ricordiamo
infatti che

\[ \Delta U = Q - W \quad\text{ ma }\quad dU = \delta Q - \delta W \]

\noindent dove $\delta$ indica una quantità dipendente dai punti attraversati.
La differenza finale $dU$ sarà sempre la stessa, ma può derivare
da contributi differenti di calore e lavoro. Nelle prossime sezioni
sostituiremo $\delta$ con $d$, perché ci limiteremo a definire una
sola trasformazione alla volta, quindi un solo tratto di percorso
ben definito. L'equazione $dU = \delta Q - \delta W$ è più generale
perché presuppone la possibilità che il gas possa compiere più di una
trasformazione per variare la propria energia interna.

Il ``percorso'' prende il nome di \textit{trasformazione termodinamica}
e studieremo ora le più semplici. Ricordiamo inoltre che le
trasformazioni che verranno mostrate sono \textit{quasistatiche}
e che quindi ogni punto può essere approssimato ad uno stato di
equilibrio differente.

\subsection{Trasformazione isocora}
Nelle trasformazioni isocore, il volume del gas rimane costante.
Segue dunque

\[ dV = 0 \Rightarrow dW = pdV = 0 \Rightarrow dU = dQ \]

\noindent Sappiamo dagli studi di calorimetria che possiamo associare al
gas un certo calore specifico $c_v$ che permette di descrivere il calore
scambiato rispetto alla variazione di temperatura. Utilizziamo per
convenienza un calore specifico \textit{molare}, anziché ricorrere
alla massa. Bisogna però specificare che tale calore specifico
vale solo a volume costante.

\[ dQ = nc_VdT \]

Da cui

\[ dU = nc_VdT \]
Dal momento che l'energia interna di un gas dipende solamente dalla
temperatura, ciò vale anche per la sua variazione. Dunque, anche
nelle trasformazioni che vedremo di seguito possiamo supporre che
valga tale relazione.

Dalla teoria cinetica abbiamo inoltre osservato che l'energia
interna di un gas equivale a $U = \frac32nRT$, da cui
$dU = \frac32nRdT$
Possiamo dunque concludere con la seguente proposizione (per i
gas ideali monoatomici):

\[ c_V = \frac{3}{2}R \]

\noindent In generale, avendo $l$ gradi di libertà vale $c_V = \frac{l}{2}R$.

\subsection{Trasformazione isobara}
Per una trasformazione isobara, la pressione rimane costante.
Diventa dunque facile calcolare il lavoro effettuato in tale
trasformazione:

\[ W_\text{isobara} = p\Delta V = nR\Delta T \]

\noindent Dalle trasformazioni isocore abbiamo concluso che $dU = nc_VdT$.
Per quanto riguarda il calore, dobbiamo introdurre un ulteriore
calore specifico $c_p$ che invece vale per trasformazioni isobare.

\[ dQ = nc_pdT \]

\noindent Riproponiamo l'equazione di stato dei gas ideali $pV = nRT$ e
consideriamo una variazione infinitesimale su entrambi i membri.

\[ d[pV] = d[nRT] \]
\[ dpV + pdV = nRdT \]
\[ pdV = dW = nRdT \]

\noindent Dunque $dU = nc_pdT - nRdT = nc_vdT$, da cui la seguente relazione.

\begin{align}
    c_p - c_v = R
\end{align}

\noindent Quella che abbiamo appena ottenuto è la cosiddetta \textit{relazione
di Mayer} (esatto, lo studentato). Dai risultati precedenti otteniamo

\[ c_p = c_v + R = \frac52R \]

Se riformuliamo la relazione di Mayer otteniamo una ulteriore relazione
molto impiegata nello studio di un'altra trasformazione che vedremo.

\[ R = c_V\left(\frac{c_p}{c_V} - 1\right) = c_V(\gamma - 1) > 0 \]

\noindent Incontreremo di nuovo il rapporto $\gamma$. Da questo risultato
risulta dunque che

\[ c_p > c_V \]

\noindent sempre (entro il nostro modello). Ma come mai è importante precisare
questa osservazione? In verità tale relazione ha un riscontro reale
e sperimentale: in una trasformazione isocora, il volume non varia;
d'altra parte in trasformazioni isobare la pressione viene mantenuta
costante e ciò significa che il gas deve esercitare tale pressione
costante durante tutal la trasformazione. Ciò comporta la necessità di un
supplemento di energia dovuto alla variazione di volume e alla costanza
della pressione. Di fatto, per aumentare l'energia interna di un
gas è necessario fornire maggior energia qualora la pressione rimanga
costante (per esempio in un contenitore chiuso da uno stantuffo mobile,
dove il gas impiega energie per sollevare lo stantuffo), mentre risulta
meno dispendioso innalzare l'energia interna di un gas a volume
costante. Da qui il motivo della disuguaglianza $c_p > c_V$.

\subsection{Trasformazioni isoterme}
Nelle trasformazioni isoterme, la temperatura non varia. Come
conseguenza, nemmeno l'energia interna varia.

\[ dU = nc_VdT = 0 \]

Segue dunque che

\[ dQ - dW = 0 \]
\[ dQ = dW = pdV = nRT\frac{dV}{V} \]
\[ Q_\text{isoterma} = \int_{i}^{f}nRT\frac{dV}{V} = nRT\ln\left(\frac{V_f}{V_i}\right) = W_\text{isoterma} \]

In una isoterma, dunque, non varia l'energia interna, perché il calore
scambiato viene ``consumato'' in una quantità uguale di lavoro.

\subsection{Trasformazioni adiabatiche}
In queste trasformazioni non avvengono scambi di calore.

\[ dU = dQ - dW = -dW \]

Come sempre, sapendo che la variazione di energia interna è esprimibile
attraverso una corrispondente variazione di temperatura legata a $c_V$,

\[ nc_VdT = -pdV \]

\[ nc_VdT = -nRTdV \]

Otteniamo dunque la seguente relazione:

\[ \frac{dT}{T} = -\frac{R}{c_V}\frac{dV}{V} \]

\[ \int_{i}^{f}\frac{dT}{T} = -\frac{R}{c_V}\int_{i}^{f}\frac{dV}{V} \]

\[ \ln\left( \frac{T_f}{T_i} \right) = -(\gamma - 1)\ln\left(\frac{V_f}{V_i}\right) \]

\[ T_i V_i^{\gamma - 1} = T_fV_f^{\gamma - 1} \]

Vale allora per le adiabatiche

\begin{align}
    TV^{\gamma - 1} = \text{const}\\
    pV^\gamma = \text{const}
\end{align}

In particolare, nel caso di trasformazioni adiabatiche che coinvolgono
gas monotatomici risulta $pV^{\frac53} = \text{const}$.

\section*{Trasformazioni cicliche}
Combinando insieme le trasformazioni semplici viste fino ad ora, è possibile
comporre trasformazioni più complesse. Tra queste, quelle più interessanti
sono forse le \textit{trasformazioni cicliche} (o cicli termodinamici), nelle
quali, data una coordinata termodinamica, il sistema ritorna nello stesso
punto dopo aver compiuto alcune trasformazioni. Seguendo i nostri presupposti,
possiamo concludere che la variazione di energia interna dei sistemi che
subiscono trasformazioni cicliche non varia per definizione:

\[ \Delta U_\gamma = 0 \quad \forall \gamma = A \to A \]

\noindent in quanto quest'ultima dipende unicamente dallo stato.

\section*{Macchine}
I cicli termodinamici caratterizzano i principi di funzionamento delle
\textit{macchine}. Il motore due-tempi dell'Husqvarna sotto
casa è una macchina; ma lo è anche un freezer. Di fatto, la termodinamica
distingue due categorie principali di macchine sulla base del verso con il
quale avvengono gli scambi di energia:

\begin{itemize}
    \item \textbf{Macchine termiche:} per queste macchine valgono
    
    \[ W,Q > 0 \]
    ovvero compiono lavoro al prezzo di un consumo di calore.

    \item \textbf{Macchine frigorifere:} vale
    
    \[ W,Q < 0 \]
    ossia esse dissipano calore al prezzo di un lavoro esterno.
\end{itemize}

Notare che la classificazione parte dal segno del lavoro, per poi
dedurre il flusso di calore mediante la definizione di trasformazione
ciclica combinata alla prima legge della termodinamica $\Delta U = 0 = Q - W$.

\subsection{Rendimento}
Sappiamo bene che è necessario un costo in termini energetici (dunque monetari,
ambientali e così via nella nostra vita economica) per far funzionare una
macchina e ottenere da essa i risultati attesi. Dal motore dell'auto è desiderabile
ottenere il massimo della distanza percorribile e della potenza sprigionabile
partendo dalla benzina pagata al pieno precedente. Ma come valutare la bontà
di un motore?

Consideriamo una macchina termica, ovvero il motore dell'auto. Il suo scopo
è produrre lavoro, cioè trasmettere forza motrice alle ruote. Per far funzionare
una macchina termica è necessario fornire calore. Dunque, possiamo valutare
l'\textit{efficienza} del motore dell'auto secondo il seguente rapporto:

\begin{align}
    \eta = \frac{W}{Q_A}
\end{align}


\noindent Dove $Q_A$ rappresenta il calore assorbito dalla macchina, o l'energia pagata
e immessa nell'auto. Possiamo riformulare la definizione di rendimento
notando che, durante il funzionamento del motore, parte del calore assorbito
$Q_A$ viene sì trasformato in lavoro, ma anche in altre forme, che racchiuderemo
nel calore totale ceduto $Q_C$. Vale dunque:

\[ \Delta U = 0 = Q - W = Q_A + Q_C - W \]

\[ W = Q_A + Q_C \]

\noindent da cui riformuliamo l'efficienza:

\begin{align}
    \eta = 1 + \frac{Q_C}{Q_A} = 1 - \frac{|Q_C|}{|Q_A|}
\end{align}

\noindent Notare che, per definizione di macchina termica, il calore
ceduto possiede segno negativo.

\subsection{Utopie}
Chiaramente, senza l'intervento di agenti esterni, non è possibile
ottenere dal nostro motore più lavoro di quanto calore è stato fornito
inizialmente

\[ Q_A < W \]

\noindent perché, come sappiamo dalla meccanica, l'energia si conserva
e non è possibile dunque crearne ``magicamente''. È però lecito chiedersi
se è possibile ottenere qualcosa come questo:

\[ Q_A \stackrel{?}{=} W \]

\noindent ovvero costruire un motore completamente efficiente. Sempre
dalla meccanica, possiamo ipotizzare che, anche solo teoricamente e
in situazioni ideali, ciò sarebbe ammesso, perché i sistemi meccanici
sono \textit{reversibili}, ovvero è sempre possibile riportarli allo
stato iniziale dopo un certo fenomeno. Non aggiungiamo altro per riservare
la sorpresa ai lettori.

In ogni caso, un buon motore dovrebbe produrre quanto più lavoro rispetto
ad un quantitativo basso di calore assorbito:

\begin{center}
    \textit{Minimo sforzo, massima resa. Spendi poco, ottieni molto.}
\end{center}

\section*{Il secondo principio della termodinamica}
Il primo principio della termodinamica ci consente di osservare che non
è possibile costruire macchine in grado di sprigionare più energia di
quella con la quale la alimentiamo. Possiamo però notare che se $\Delta U = 0$
allora $Q = W$. Ciò può far sperare che una macchina possa comunque
convertire l'intera energia fornita in lavoro, senza alcuna dispersione.
Tuttavia, ciò è talmente difficile da sembrare impossibile (vedremo che
effettivamente lo è). Parimenti è sperimentalmente evidente che il
calore non fluisce \textit{mai spontaneamente} da una sorgente fredda
da una calda. Quest'ultima osservazione in particolare è la base che
supporta il secondo principio della termodinamica, del quale sono date
due formulazioni ben note:

\begin{itemize}
    \item \textbf{Enunciato di Kelvin-Plank}
    
    \begin{center}
        \textit{È impossibile realizzare una macchina il cui unico risultato\\
        sia quello di trasformare calore, da una sorgente, interamente in lavoro.}
    \end{center}

    \item \textbf{Enunciato di Clausius}
    
    \begin{center}
        \textit{È impossibile realizzare un processo il cui unico risultato\\
        sia quello di trasferire calore da un corpo ad uno più caldo.}
    \end{center}
\end{itemize}


\noindent Purtroppo per la nostra umanità (e probabilmente per tutte le civiltà avvenire),
non è possibile realizzare macchine perfette, ovvero dal rendimento del
$100\%$, perché questo è impedito dal secondo principio, in particolar modo
dalla formulazione di Kelvin-Plank.

Entrambe le formulazioni sono date per vere, ma possiamo tuttavia dimostrarne
l'equivalenza.

\section*{Esperienza di Carnot}
Qual è la macchina più efficiente in assoluto? Oppure, in quali condizioni
una macchina può ottenere il rendimento massimo? Queste sono domande che
Carnot si pose ai suoi tempi.

\section*{Esperienza di Clausius}
Altra figura importante in termodinamica è il tedesco Rudolf Clausius,
che estese gli studi importanti di Carnot ad un livello superiore.

\subsection{Teorema di Clausius}
Abbiamo appena visto che una delle relazioni ottenute da Carnot è la
somma di rapporti $\frac{Q_1}{T_1} + \frac{Q_2}{T_2} \leq 0$. Ciò vale
per tutte le macchine, reversibili o no, che operano tra due sorgenti
$T_1 < T_2$. Avere però solamente due sorgenti è piuttosto limitante,
perché nella realtà le macchine operano su cicli ben più complessi e
dunque è come se operassero tra più di due sorgenti. Clausius ebbe così
la brillante idea di generalizzare le conclusioni di Carnot per un
ciclo termodinamico qualsiasi.

Supponiamo di osservare un cilo motore qualsiasi (la nocciolina motrice,
che carina). Questo ciclo
attraversa gradi di temperatura molteplici. Possiamo però effettuare
una approssimazione della curva del ciclo immaginando di percorrerla
mediante piccole trasformazioni adiabatiche e isoterme. Se estendiamo
queste piccole trasformazioni, otterremo
una suddivisione dell'area interna del nostro ciclo in tanti piccoli
cicli di Carnot. Per allora il ciclo $i$-esimo, vale la relazione

\[ \frac{Q_{1,i}}{T_{1,i}} + \frac{Q_{2,i}}{T_{2,i}} \leq 0 \]

\noindent È allora immediato osservare che sommando questa quantità
per ogni ciclo interno all'area del ciclo vale

\[ \sum_{i} \left(\frac{Q_{1,i}}{T_{1,i}} + \frac{Q_{2,i}}{T_{2,i}}\right) \leq 0\]

\noindent Questa relazione non ci dice ancora molto di straordinario.
Possiamo però notare due cose: i tratti adiabatici non coinvolgono
scambi di calore per definizione, e dunque possiamo ignorarli nel
nostro calcolo, perché il rapporto calore-temperatura è sempre nullo
durante questo tipo di trasformazione; esistono molti tratti isotermi
che vengono percorsi in versi opposti. Questi sono i tratti \textit{interni}
alla nostra curva origniale ed essendo percorsi in versi opposti non
introducono ulteriori contributi calorici. Gli unici tratti di nostro
interesse, dunque, sono quelli delle trasformazioni isoterme \textit{ai
bordi} del ciclo. Possiamo allora riformulare la sommatoria unendo
solo questi rapporti. Supponendo che il tratto $j$-esimo sia una isoterma
che costituisce un pezzo di bordo della curva originale, otteniamo

\[ \sum_{j} \frac{Q_j}{T_j} \leq 0 \]

\noindent o per i più duri e cazzuti come noi, vere allieve e veri allievi
del Maestro Iuppa,

\begin{align}
    \oint \frac{dQ}{T} \leq 0
\end{align}

\noindent che è proprio la tesi del teorema di Clausius. L'integrale
prende il nome di \textit{integrale di Clausius}, giusto per sottolineare
che se un qualche oggetto matematico porta il nome di una persona, esso è più
importante di quanto si possa credere. Inoltre, come sempre, l'uguaglianza
vale solamente per trasformazioni reversibili.

\subsection{Entropia}
Supponiamo di avere un ciclo reversibile $\gamma$ qualsiasi. Essendo
reversibile, possiamo concludere che

\[ \oint_\gamma \frac{dQ}{T} = 0 \]

\noindent Selezioniamo due punti distinti su questo ciclo, $A$ e $B$
e scomponiamo l'integrale di Clausius nei due percorsi $\alpha$ e
$\beta$:

\[ \int_{A,\alpha}^{B} \frac{dQ}{T} + \int_{B,\beta}^{A} \frac{dQ}{T} = 0 \]

\noindent Trattandosi di un ciclo reversibile, vale la proprietà
antisimmetrica dell'integrale. Giungiamo dunque alla seguente conclusione:

\[ \int_{A,\alpha}^{B} \frac{dQ}{T} = \int_{A,\beta}^{B} \frac{dQ}{T} \]

\noindent Dunque, qualsiasi sia il percorso tra $A$ e $B$, luguaglianza
precedente vale sempre per percorsi reversibili. Vi è dunque una dipendenza
di una certa quantita dai soli stati iniziale e finale. Definiamo dunque,
come abbiamo fatto per l'energia potenziale, la \textit{differenza di
entropia tra due stati termodinamici:}

\[ \Delta S_{AB} = S_B - S_A \stackrel{\text{def}}{=} \int_{A,\text{rev}}^{B} \frac{dQ}{T} \]

\noindent L'entropia è a tutti gli effetti una funzione di stato. Possiamo
inoltre ricavare le seguenti osservazioni:

\begin{itemize}
    \item La somma dell'entropia di due sottosistemi corrisponde all'entropia
    del sistema complessivo. Da questa osservazione è possibile supporre
    che l'universo abbia una propria entropia, derivante da tutti i suoi
    sottosistemi, conosciuti e non.

    \item Si tratta di una grandezza estensiva. Ha significato solo se si
    considera l'intero sistema.

    \item Per calcolare le differenze di entropia nelle trasformazioni
    reversibili, è sufficiente essere furbi utilizzando le trasformazioni
    reversibili più comode dal punto di vista computazionale, perché
    l'entropia è una funzione di stato e non dipende dal percorso.
\end{itemize}

\subsection{Variazione di entropia nelle trasformazioni reversibili elementari}
\begin{itemize}
    \item Isoterma:
\end{itemize}


\subsection{Il teorema dell'entropia}
Consideriamo due trasformazioni termodinamiche che collegano due punti
$A$ e $B$ descritti da coordinate termodinamiche. Tuttavia, il primo,
$\alpha$, è irreversibile, mentre $\beta$, il secondo, è reversibile.
Unendo $\alpha$ e $\beta$ (invertito), otteniamo un ciclo per il quale
vale la seguente disuguaglianza, che segue dal teorema di Clausius

\[ \oint_\gamma \frac{dQ}{T} < 0 \]

\noindent Perché il tratto $\alpha$ rende l'intera trasfomrazione irreversibile.
Come per la definizione di variazione di entropia, spezziamo l'integrale
di Clausius:

\[ \oint_\gamma \frac{dQ}{T} = \int_{A,\alpha}^{B} \frac{dQ}{T} + \int_{B,\beta}^{A} \frac{dQ}{T} < 0 \]

\noindent Da cui, sapendo che $\beta$ è reversibile,

\[ \int_{A,\alpha}^{B} \frac{dQ}{T} < \int_{A,\beta}^{B} \frac{dQ}{T} = \Delta S_{AB} \]

Il fatto che la variazione di entropia dipenda solo dagli stati
iniziale e finale, è possibile concludere con la seguente generalizzazione

\begin{align}
    \Delta S_{AB} \geq \int_{A,\text{qualsiasi}}^{B} \frac{dQ}{T}
\end{align}

\noindent che è l'enunciato del teorema dell'entropia (come sempre,
l'uguaglianza vale solamente per trasformazioni reversibili). Dal
teorema segue un'osservazione inquietante. Per un sistema isolato,
non vi è alcuno scambio di calore con l'esterno, qualunque siano le
trasformazioni che stanno avvenendo al suo interno, reversibili oppure
no. Dunque $dQ = 0$ e l'integrale è sempre nullo.

\[ \Delta_{AB} \geq \int_{A}^{B} \frac{dQ}{T} = 0 \qquad \Delta_{AB} \geq 0 \]

\noindent ovvero

\[ S_B \geq S_A \]

\noindent Segue che per il nostro sistema isolato l'entropia non può mai
diminuire; essa può al più rimanere costante. Ma la cosa più sconcertante
è che anche quest'ultima prospettiva è pressoché un miraggio: dovremmo
riprodurre trasformazioni \textit{sempre reversibili}.

\subsection*{Entropia e tempo}