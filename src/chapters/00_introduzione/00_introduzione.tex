\chptr{Introduzione alla Fisica}
\marginpar{\minitoc}

\section{Definizione e scopi della fisica}

Si possono formulare definizioni diverse riguardo la disciplina scientifica
della fisica, come la seguente:

\vspace{8pt}
\begin{tcolorbox}[colback = yellow!30, colframe = yellow!30!black, title = {Fisica}]
La fisica è lo studio quantitativo delle leggi fondamentali della natura, cioè
delle leggi che governano tutti i fenomeni naturali dell'universo.

Una legge fisica (o principio) è una regolarita' della natura esprimibile in forma
matematica, ma anche una verità non dimostrabile che tuttavia non contraddice i
fenomeni osservabili dell'esperienza.
\end{tcolorbox}
\vspace{5pt}

La fisica si avvale del \textbf{metodo scientifico}, secondo cui la natura deve
essere interrogata per vie sperimentali, facendosi guidare da \textbf{ipotesi} e
modelli teorici. Una particolarita' di questo metodo è la capacita' di isolare
un certo fenomeno che si intende studiare, tralasciando (si usera' spesso il
termine \textit{trascurare}) certi aspetti ritenuti non rilevanti in modo da
scoprire quelle regolarita' dalle quali potrebbe essere dedotta una certa'
relazione matematica.

Il ruolo della matematica è di fornire un linguaggio formale per descrivere
quantitativamente i fenomeni osservati e costruire modelli utili alla loro
trattazione.



\section{Grandezze fisiche}
La fisica è una scienza quantitativa, ovvero essa si occupa di caratteristiche
e proprieta' del mondo che possono essere misurate e quantificate: le cosiddette
grandezze fisiche. Esempi di grandezze fisiche sono la lunghezza, la massa, la
temperatura, la durata temporale e cosi' via.

\vspace{8pt}
\begin{tcolorbox}[colback = yellow!30, colframe = yellow!30!black, title = {Grandezza fisica}]
Una grandezza fisica è una caratteristica di un oggetto o di un fenomeno che puo'
essere misurata in termini quantitativi (oltre che oggettivi, ovvero indipendentemente
dalle sensazioni personali degli individui).
\end{tcolorbox}
\vspace{5pt}

È implicito, intuitivamente, il concetto di \textbf{misura}. Misurare una grandezza
fisica significa confrontarla con una grandezza ``campione'', detta \textbf{unita'
di misura}, e stabilire quante volte l'unita' di misura è contenuta nella
grandezza data. Il valore numerico ottenuto è la misura della grandezza e deve
essere sempre accompagnato dall'unita' di misura.
In altre parole, la \textbf{misura} non è altro che un \textit{rapporto} tra la
grandezza che si intende misurare e la grandezza campione scelta convenzionalmente
per tale scopo.

Mostriamo un esempio: supponiamo di voler misurare la lunghezza di qualsiasi cosa
in ``chiavette USB'' (si potrebbe argomentare circa quale chiavetta si stia
impiegando e quale posizione la chiavetta debba assumere durante la misura.
Supponiamo qui che la chiavetta sia posta in verticale, senza perderci in ulteriori
dettagli). Decidiamo poi di misurare l'altezza di una porta—anche qui, non
specifichiamo quale porta—utilizzando l'unita' appena scelta. Supponiamo quindi
di aver registrato il seguente dato:
\[ H = 20 \text{ chiavette USB} \]
Notare come siano stati specificati:
\begin{itemize}
    \item Un nome per l'oggetto che si intendeva misurare, $H$, ovvero l'altezza
    della porta.
    \item Il valore numerico individuato, 20.
    \item Una affermazione per legare il nome e il dato, = (``corrisponde a'', ``è
    uguale a'')—caratteristica che peraltro si trova anche nei linguaggi di
    programmazione.
    \item L'unita' di misura, chiavette USB.
\end{itemize}
Tuttavia, tale misurazione non è stata affatto ``sincera'': non vi è la
garanzia del fatto che il valore registrato sia esatto. La prossima sezione
trattera' questo problema, ovvero quello dell'\textit{incertezza}.


\section{Incertezza}
Idealmente, si vorrebbe impiegare, grazie alle misure, numeri puntuali ed esatti.
In altre parole, dei numeri con una precisione indefinita, aventi un numero
illimitato di cifre decimali e non.

Ma quando si effettua una misura di una grandezza, il risultato ottenuto è noto solo
con una certa precisione. Riprendendo l'esempio della chiavetta USB, è impossible
misurare con certezza tutte le lunghezze, in quanto non multipli esatti della
chiavetta stessa: ci sara' sempre un certo margine di ``un pezzo di chiavetta'',
minore dell'unita' prescelta. Ma al di sotto di quella unita' non è possibile
fornire alcuna garanzia sulla puntualita' del dato. In altre parole, la
\textit{sensibilità}\footnote{La più piccola variazione della grandezza che lo
strumento è in grado di rilevare.} dello strumento è uno dei limiti alla precisione
della misura.

\vspace{8pt}
\begin{tcolorbox}[colback = yellow!30, colframe = yellow!30!black, title = {Cifre significative del risultato di una misura}]
Le cifre significative del risultato di una misura sono le cifre note
con certezza e la prima cifra incerta. In altre parole, esse sono le cifre che si
possono controllare con lo strumento impiegato nella misura.
\end{tcolorbox}
\vspace{5pt}

Ad esempio, il valore corrispondente alla lunghezza di una barca $L = 10,5$ m
possiede tre cifre significative, che non equivale a $10,50$ m. Il secondo dato,
infatti, dichiara che la misurazione è stata possibile controllando le cifre
fino al centimetro. $L = 0,002$ possiede solo una cifra significativa, perché
in genere si ignorano gli zeri a sinistra della prima cifra significativa diversa
da zero. Possono essere ambigui valori come $L = 2500 \text{ m}$: quali zeri sono
cifre significative? Come vedremo tra poco, è utile esprimere questi valori in
notazione scientifica per eliminare ambiguità.

Vi potrebbero anche essere errori dovuti a imprecisioni introdotte nell'utilizzo
degli strumenti di misura. Questo errore deve tuttavia essere quantificato ed ogni
misura ne è affetta (comprese quelle che non la riportano).

\vspace{8pt}
\begin{tcolorbox}[colback = yellow!30, colframe = yellow!30!black, title = {Risultato della misura di una grandezza}]
Il risultato della misura di una grandezza è sempre un'approssimazione
accompagnata da una certa incertezza, ovvero un \textbf{valore attendibile}
e un \textbf{errore assoluto} (o semplicemente \textit{incertezza}).
\[ x = \overline{x} \pm e_x  \]
\end{tcolorbox}
\vspace{5pt}

Questo risultato non è quindi altro che un intervallo in cui il valore reale
della misura si trova. Ci limiteremo agli errori relativi a singole misure,
nelle quali $x$ corrisponde al valore misurato e $e_x$ la sensibilità dello
strumento. Di conseguenza, possiamo ora correggere il risultato della misura
effettuata in chiavette USB:
\[ H = (20 \pm 1) \text{ chiavette USB} \]

\section{Notazione scientifica e ordini di grandezza}
Unità di misura come il metro e il kologrammo sono comode nella vita di tutti i
giorni, ma rappresentano quantità enormi su scala atomica e subatomica e quantità
minuscole su scala astronomica e cosmica. Conseguenza di ciò è che alcune misure
possono essere espresse da numeri ``scomodi''. Considerando solo valori attendibili,
la massa dell'atomo di idrogeno è circa
\[ m_H = 0,000 000 000 000 000 000 000 000 001 67 \text{ kg} \]
mentre la massa della Terra è
\[ m_T = 5 970 000 000 000 000 000 000 000 \text{ kg} \]
È pressoché evidente il motivo di tale scomodità: la notazione è di difficile
trattazione. Viene dunque in aiuto la \textbf{notazione scientifica}, ovvero una
notazione numerica che permette di contrarre rappresentazioni estese impiegando
potenze di 10. Nella notazione scientifica, ogni numero è scritto come prodotto
di due fattori:
\begin{itemize}
    \item Un numero decimale $x:x\in R, 1\leq x < 10$\footnote{In realtà, questa notazione corrisponde alla variante ``ingegneristica''. Esiste anche una notazione che prevede che il valore espresso $x$ sia $0\leq x < 1$.}.
    \item Una potenza di 10, con esponente intero.
\end{itemize}
Pertanto, le misure precedenti si possono esprimere in notazione scientifica come
segue:
\[ m_H = 1,67 \cdot 10^{-27} \text{ kg} \]
\[ m_T = 5,97 \cdot 10^{24} \text{ kg}\]
Notare come la notazione sia in grado di eliminare ambiguità sul numero di cifre
significative: ora sappiamo che la massa della Terra è stata calcolata fino a
tre cifre significative e non 25.

Non sempre è necessario calcolare esattamente il valore di una certa grandezza.
Talvolta basta averne solo un'idea approssimata. Supponiamo, ad esempio, che sia
sufficiente sapere se una certa massa vale all'incirca 1 grammo oppure 1
ettogrammo. In questo caso, possiamo accontentarci di stimare il valore della
massa con un'accuratezza di un fattore 10, cioè di calcolare il suo ordine di
grandezza.

\vspace{8pt}
\begin{tcolorbox}[colback = yellow!30, colframe = yellow!30!black, title = {Ordine di grandezza}]
L'ordine di grandezza di un numero è la potenza di 10 più vicina a quel numero.
\end{tcolorbox}
\vspace{5pt}

Per determinare l'ordine di grandezza di un numero occorre quindi esprimerlo in
notazione scientifica—prodotto di un numero decimale compreso tra 1 e 10 e di
una potenza di 10—e poi approssimare il valore alla potenza di 10 più vicina.
In particolare:
\begin{itemize}
    \item Se il numero decimale è minore di 5, si mantiene l'esponente della
    potenza. Ad esempio:
    \[ 3,6 \cdot 10^2 \to \text{ Ordine di grandezza } 10^2 \]
    \[ 4,2 \cdot 10^{-3} \to \text{ Ordine di grandezza } 10^{-3} \]

    \item Se il numero decimale è maggiore di 5, si somma +1 all'esponente della
    potenza. Ad esempio:
    \[ 9 \cdot 10^2 \approx 10 \cdot 10^2 \to \text{ Ordine di grandezza } 10^3 \]
    \[ 8,1 \cdot 10^{-12} \approx 10 \cdot 10^{-12} \to \text{ Ordine di grandezza } 10^{-11} \]
\end{itemize}


Sono stati definiti dei prefissi stadard per certi ordini di grandezza notevoli,
cioè quelli che, escludendo la potenza nulla, rappresentano multipli di tre.
Utilizzando questi prefissi, di fianco all'unità di misura adottata, si contrae
ancora di più la notazione scientifica, sottointendendo un certo ordine di
grandezza.

\marginpar{
    \footnotesize
    %\hspace*{-0.5cm}
    \begin{tabular}{c|c|c}
        Potenza   & Simbolo & Prefisso\\
        \hline
        $10^{12}$  & T       & Tera\\
        $10^{9}$   & G       & Giga\\
        $10^{6}$   & M       & Mega\\
        $10^{3}$   & k       & kilo\\
        $10^{-3}$  & m       & milli\\
        $10^{-6}$  & $\mu$   & micro\\
        $10^{-9}$  & n       & nano\\
        $10^{-12}$ & p       & pico\\
    \end{tabular}
}